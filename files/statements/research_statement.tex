% Options for packages loaded elsewhere
\PassOptionsToPackage{unicode}{hyperref}
\PassOptionsToPackage{hyphens}{url}
%
\documentclass[
]{article}
\usepackage{amsmath,amssymb}
\usepackage{iftex}
\ifPDFTeX
  \usepackage[T1]{fontenc}
  \usepackage[utf8]{inputenc}
  \usepackage{textcomp} % provide euro and other symbols
\else % if luatex or xetex
  \usepackage{unicode-math} % this also loads fontspec
  \defaultfontfeatures{Scale=MatchLowercase}
  \defaultfontfeatures[\rmfamily]{Ligatures=TeX,Scale=1}
\fi
\usepackage{lmodern}
\ifPDFTeX\else
  % xetex/luatex font selection
\fi
% Use upquote if available, for straight quotes in verbatim environments
\IfFileExists{upquote.sty}{\usepackage{upquote}}{}
\IfFileExists{microtype.sty}{% use microtype if available
  \usepackage[]{microtype}
  \UseMicrotypeSet[protrusion]{basicmath} % disable protrusion for tt fonts
}{}
\makeatletter
\@ifundefined{KOMAClassName}{% if non-KOMA class
  \IfFileExists{parskip.sty}{%
    \usepackage{parskip}
  }{% else
    \setlength{\parindent}{0pt}
    \setlength{\parskip}{6pt plus 2pt minus 1pt}}
}{% if KOMA class
  \KOMAoptions{parskip=half}}
\makeatother
\usepackage{xcolor}
\setlength{\emergencystretch}{3em} % prevent overfull lines
\providecommand{\tightlist}{%
  \setlength{\itemsep}{0pt}\setlength{\parskip}{0pt}}
\setcounter{secnumdepth}{-\maxdimen} % remove section numbering
\ifLuaTeX
  \usepackage{selnolig}  % disable illegal ligatures
\fi
\IfFileExists{bookmark.sty}{\usepackage{bookmark}}{\usepackage{hyperref}}
\IfFileExists{xurl.sty}{\usepackage{xurl}}{} % add URL line breaks if available
\urlstyle{same}
\hypersetup{
  pdftitle={Research statements},
  pdfauthor={Peter Iyer},
  hidelinks,
  pdfcreator={LaTeX via pandoc}}

\title{Research statements}
\author{Peter Iyer}
\date{October 31, 2023}

\begin{document}
\maketitle

My research seeks to advance our understanding of the intersection of
macroeconomics and labour economics. In particular, I am interested in
how labour market policies affect labour market outcomes. Aggregate
productivity has been stagnant for a few decades
now\footnote{https://press.princeton.edu/books/paperback/9780691175805/the-rise-and-fall-of-american-growth
  and
  https://www.brookings.edu/wp-content/uploads/2021/03/BPEASP21\_Gordon\_conf-draft.pdf}\footnote{https://www2.deloitte.com/us/en/insights/economy/behind-the-numbers/decoding-declining-stagnant-productivity-growth.html}\footnote{https://www.cairn.info/revue-de-l-ofce-2018-3-page-37.htm}\footnote{https://www.bls.gov/opub/mlr/2021/article/the-us-productivity-slowdown-the-economy-wide-and-industry-level-analysis.htm}.
What role do labour market polices play in addressing this concern?
Furthermore, how does the heterogeneous behavioral response to labour
market policies at the micro-data level give rise to the macro economic
outcomes we observe.

\textbf{Labour market policies}: The labor market is witness to various
kinds of imperfections that impede competitive outcomes. From search
costs to monopsonies. One of the most far-reaching developments in
labour economic research has been the use of quasi-experimental designs
to evaluate policy outcomes and effectiveness.

Minimum wages were first introduced in the US at the state level in late
19th century at the state-level but would get struck down fairly
quickly. The first federal minimum wage law was introduced in 1938, and
since the 2000s the states have been the primary drivers of minimum wage
policy. While the literature tends to focus on the demand effects of
these policies, little work has been done to address the supply response
to minimum wages. In \emph{Effects of minimum wages on reservation
wages} I use quasi-experimental methods along with a little known, but
nationally relevant data source to find minimum wage increases are
associated with a rise in reservation wages. This increase rises with
the increase in minimum wages and the response elasticity varies between
0.44 to 0.82 in magnitude.

In \emph{Effects of UI generosity on unemployment} I study the impact of
the CARES act using Burning Glass data with an extended
Diamond-Mortensen-Pissarides model (first developed in
@gallant2020temporary). This model accounts for the heterogeneity among
the unemployed on the basis of their attachment to their employers. This
heterogeneity became pronounced during the 2020 pandemic recession which
saw some workers get furloughed, and others laid-off or exit the labour
force. Failure to account for this heterogeneity can been seen in the
2020 Beveridge curve which is downward sloping and inelastic, unlike the
Beveridge curve in most recessions, which is backwards bending. Using
this model I compare the unemployment in states that terminated enhanced
UI early with the states that didn't and I find that states that
terminated the augmented UI early saw a rise in aggregate search effort.

In a separate project with \href{https://www.hyunjaekang.com/}{Jay
Kang}, I use panel data from the UK to study the effects of university
tuition hikes on major choices by young Britons. In 2017, the 2010 UK
Coalition government introduced policy changes to higher education
funding that effectively tripled university tuition fees from £3000 to
£9000 per year. The estimate of debt from the Institute of Fiscal
Studies for students following this policy change is over £44000. Thus,
we plan to study the effects of this policy shock on the human capital
investment decisions made by young Britons.

\textbf{Macroeconomic outcomes} Beyond the microeconomic realm, my
research extends to the macroeconomic consequences of diverse labour
market policies. I am interested in exploring how the aggregate effects
of these policies resonate through the economy, affecting employment,
and productivity.

I am working on incorporating some of my previous findings to simulate
policy experiments. I'd like to use a survival analysis model with a
job-search model on my reservation wage data to study the effect of a
\$15 minimum wage on unemployment spell duration in \emph{Spell duration
response to minimum wage shocks}.

The common theme underlining my work is a strong sense of empirical
motivation. I combine dynamic modelling techniques and rich data, I aim
to provide nuanced insights into the causal relationships between labour
market polices, microeconomic responses, and macroeconomic outcomes.

\end{document}
