% Options for packages loaded elsewhere
\PassOptionsToPackage{unicode}{hyperref}
\PassOptionsToPackage{hyphens}{url}
%
\documentclass[
]{article}
\usepackage{amsmath,amssymb}
\usepackage{iftex}
\ifPDFTeX
  \usepackage[T1]{fontenc}
  \usepackage[utf8]{inputenc}
  \usepackage{textcomp} % provide euro and other symbols
\else % if luatex or xetex
  \usepackage{unicode-math} % this also loads fontspec
  \defaultfontfeatures{Scale=MatchLowercase}
  \defaultfontfeatures[\rmfamily]{Ligatures=TeX,Scale=1}
\fi
\usepackage{lmodern}
\ifPDFTeX\else
  % xetex/luatex font selection
\fi
% Use upquote if available, for straight quotes in verbatim environments
\IfFileExists{upquote.sty}{\usepackage{upquote}}{}
\IfFileExists{microtype.sty}{% use microtype if available
  \usepackage[]{microtype}
  \UseMicrotypeSet[protrusion]{basicmath} % disable protrusion for tt fonts
}{}
\makeatletter
\@ifundefined{KOMAClassName}{% if non-KOMA class
  \IfFileExists{parskip.sty}{%
    \usepackage{parskip}
  }{% else
    \setlength{\parindent}{0pt}
    \setlength{\parskip}{6pt plus 2pt minus 1pt}}
}{% if KOMA class
  \KOMAoptions{parskip=half}}
\makeatother
\usepackage{xcolor}
\setlength{\emergencystretch}{3em} % prevent overfull lines
\providecommand{\tightlist}{%
  \setlength{\itemsep}{0pt}\setlength{\parskip}{0pt}}
\setcounter{secnumdepth}{-\maxdimen} % remove section numbering
\ifLuaTeX
  \usepackage{selnolig}  % disable illegal ligatures
\fi
\IfFileExists{bookmark.sty}{\usepackage{bookmark}}{\usepackage{hyperref}}
\IfFileExists{xurl.sty}{\usepackage{xurl}}{} % add URL line breaks if available
\urlstyle{same}
\hypersetup{
  pdftitle={Teaching statement},
  pdfauthor={Peter Iyer},
  hidelinks,
  pdfcreator={LaTeX via pandoc}}

\title{Teaching statement}
\author{Peter Iyer}
\date{October 31, 2023}

\begin{document}
\maketitle

%\{::options parse\_block\_html=``true'' /\} 
My philosophy when it comes
to teaching is rooted the belief that education is a transformative and
empowering experience. I consider it a duty of mine to foster critical
thinking, creativity, and a passion for the subject when I take on the
role of an educator. Based on my own experience as a student I try to be
sympathetic the myriad of challenges students face within and outside
the classroom. Furthermore, I aim to create an inclusive and
collaborative environment where students learn from the material and
each other.

I owe a debt of gratitude to my colleagues and past instructors. One of
my teachers once told me that stress is deeply antithetical to learning
outcomes. I prioritise student-centered learning by tailoring my
lecturing methods to accommodate their various learning styles. Students
who don't feel comfortable speaking in class are encouraged to come to
office hours or send me an email. I hold regular review classes where
students are encouraged to discuss confusing topics, and are free to
email their questions beforehand, if they'd rather stay anonymous.
Furthermore, since some students take some classes as a requirement, I
try to get them interested by bringing up contemporary topics and
discussing how they relate to the material they are learning. The most
recent example that comes to mind is when my class discussed a clip in
which
\href{https://www.youtube.com/watch?v=psSYiidw-v0&pp=ygURam9obiBzdGV3YXJ0IGRlYnQ\%3D}{John
Stuart asked Thomas Hoenig why the US debt couldn't be resolved by
minting a trillion dollar coin?.} It is a truism to say that today's
students will be the leaders of tomorrow. Thus I try to bridge the gap
between theory and practice by incorporating real world examples, case
studies, and practical applications into my curriculum and show how what
the students are learning are relevant to their lives and careers.

Creating an inclusive and respectful environment is one of the
cornerstones of my teaching philosophy, and I recognize the need to
incorporate diverse perspectives and experiences in enriching the
learning process. After all, the value of a good education is that like
a long trip abroad, it challenges our world views and dislodges us from
the mundane. From the world of business, to the world of arts and even
academia, the demand for people with novel and interesting insights far
outstrips the supply. An environment that harnesses the diversity of the
students, their backgrounds, and their rich experiences is essential to
addressing this gap.

I don't just consider tests to be a measure of knowledge but a tool to
learning as well. I return grades quickly and make sure that students
feel comfortable talking about their performance by holding office hours
that agree with everyone's schedules. I make sure to communicate the
expectations for the exams and give similar problems in the assignments
so the students are never caught by surprise.

I am dedicated to continuous development to stay informed about
pedagogical methods, emerging technologies, and advancements in my
field. This commitment allows me to bring the most up-to-date knowledge
to the classroom, and ensuring my students are at the cutting edge of
the subject. My teaching philosophy is centered on empowering students
to become critical thinkers, lifelong learners, and responsible
contributors to society. I am passionate about creating an inclusive and
engaging learning environment that prepares students for academic and
professional success.

\end{document}
